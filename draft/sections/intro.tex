\vspace{-1em}
\section{Introduction}

People care a lot about the capability of phone cameras and we still have problems while taking photos using phones.
Phone cameras of most brands these days are still limited by the basic cost and the size.~\cite{autofocus}
Some new types of phones from Apple, Samsung, LG and Google Pixel are using new methods like phase contrast and laser detection to get a more accurate distance between the camera and the object.
These methods surely improves the speed of focusing while capturing photos, at the cost of better hardware and potential slight errors.

Most phones still use traditional cameras, which includes a motor for lens group moves and hill climbing (trial-and-error) for focusing.
The lens group starts from a position, moving towards one direction at a large step and compare the focusing effect with the previous capture.
Once the photo quality decreases, the lens group moves towards the reverse direction, with a smaller step, until the step is at the minimum.

There are many previous works discussing how we should tell a photo is well-focused.
These focus functions mostly are based on derivatives, which directly show how sharp an edge is.
This indicates whether a photo is clear and not blurred.
Some methods pick all the pixels of the photo and sum up the derivatives at each pixels as the final value for focus evaluation.
Other methods pick only part of them and choose the orientation of the picked pixels respectively.

The traditional hardware and methods indicate that the speed of focusing is always slow but the outcome is fairly good.
However, there are problems when we need to speed up the photographing process, especially when we need to collect a large amount of photos or quickly collect most photos at a relatively fixed distance.
To save some time, we present \sysname as a brandnew solution, which greatly improves the photo capturing speed while the photo quality loss is little.

Our design is pretty simple.
Instead of having the lens group moving back and forth several rounds, we move in only 1 direction, collect all the photos at a larger step.
The step value should be smaller than the starting step of hill climbing but slightly larger than the minimal step in hill climbing.
This way, we may lose the accuray to some extent, but we save time by taking fewer photos and calculating focus function fewer times as well.

We implement our own version of hill climbing and 2 new versions of \sysname, all in Python.
We collect 5 groups of photos as the input for the 3 algorithms.
The object distance of these groups range from 5 centimeters to 1 meter.
The results show that v1 of \sysname has a slightly faster speed than traditional hill climbing and surprisingly better focus effect in some cases.
v2 of \sysname has much faster focusing speed with a boost up to 50.3\% and 39.7\% on average.
The quality loss is pretty low, which is less than 18\% in the worst case and 8.1\% on average.
